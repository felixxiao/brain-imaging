\section{A Global Solution to Mean Adjacent Within Edge}

We want to find a partitioning of graph $G(V,E)$ into $K$ components
so as to minimize the Adjacent-Score (\ref{adjacent-score})
\[ \frac{1}{K} \sum_{V \in \mathcal{P}_K}
   \frac{1}{|E_{V,V}|} \sum_{(i,j) \in E_{V,V}} A_{i,j}
\]

Let $m = |E|$, the number of edges in the graph. Assign each edge an
index in $\{1, ..., m\}$ and 
let $a$ be an $m$-dimensional vector whose $j$th entry is the weight of
the $j$-indexed edge. Since distance correlation is between 0 and 1, so
are the entries of $a$.

For $k = 1, ..., K$, let $z_k \in \{0, 1\}^m$ be a 0-1 vector with
$j$th entry satisfying
\[ z_{jk} = \begin{cases}
  1 & \text{if both endpoints of edge } j \text{ are in } V_k \\
  0 & \text{otherwise}
\end{cases} \]

If $z_1, ..., z_K \in \{0, 1\}^m$ describe a valid partitioning, then
they must satisfy $\sum_{k=1}^K z_{jk} \leq 1$ for all edges $j$.
However, the converse is not true, since this contraint still allows two
edges sharing an endpoint to be within different parcels.

To prevent that from happening, we introduce the assignment matrix
$X \in \{0, 1\}^{n \times m}$ with entries
\[ x_{i,k} = \begin{cases}
  1 & \text{if vertex } i \in V_k \\
  0 & \text{otherwise}
\end{cases} \]
and three constraints
\begin{align*}
1 + z_{jk} &\geq x_{hk} + x_{ik} \\
z_{jk} &\leq x_{hk} \\
z_{jk} &\leq x_{ik}
\end{align*}
for all $j = 1, ..., m$, $k = 1, ..., K$, where $(h,i)$ are the two
endpoints of edge $j$. If we constrain $X$ to be binary then the three
above constraints are equivalent to:
\[ z_{jk} = \begin{cases}
  1 & \text{if } x_{hk} = 1 \text{ and } x_{ik} = 1 \\
  0 & \text{otherwise} \\
\end{cases}\]
For the $X$, we only need to ensure every vertex is in a parcel and every
parcel has at least one vertex (or some other specified minimum):
\[ \sum_{k=1}^K x_{ik} = 1 \]
\[ \sum_{i=1}^n x_{ik} \geq 1 \]
where the equation holds for all $i$ and the inequality for all $k$.

Hence the following optimization problem finds a valid partition that
maximizes adjacent-score. This is an instance of generalized fractional
linear programming. Let $e_m$ denote a vector of $m$ ones.

\begin{center}
\bgroup
\def\arraystretch{1.5}
\begin{tabular}{l l l}
maximize   & $\dfrac{a^T z_1}{e_m^T z_1} + \cdots +
              \dfrac{a^T z_K}{e_m^T z_K}$ \\ \\
subject to 
           & $\begin{cases}
                 1 + z_{jk} \geq x_{hk} + x_{ik} \\
                 z_{jk} \leq x_{hk}             \\
                 z_{jk} \leq x_{ik}             \\
             \end{cases}$
           & $j = 1, ..., m$, $k = 1, ..., K$, $(h,i) = j$ \\
           & $\sum_{j=1}^m z_{jk} \geq 1$ & $k = 1, ..., K$ \\
           & $\sum_{k=1}^K x_{ik} = 1$ & $i = 1, ..., n$ \\
           & $\sum_{i=1}^n x_{ik} \geq 1$ & $k = 1, ..., K$ \\
           & $X \in \{0, 1\}^{n \times K}$
\end{tabular}
\egroup
\end{center}

Following the result in \cite{Li:94} we derive an equivalent Mixed
Binary Linear Program to the above.
Substitute $y_k = \dfrac{1}{e_m^T z_k}$ for each $k$, which amounts to
introducing a new variable $y \in \R^K$ and non-linear constraints
\[ e_m^T z_k y_k = 1 \]
The key theorem in \cite{Li:94} uses the fact that $z_k$ is binary to
linearize this contraint by introducing another variable $w_{jk}$ and
using linear constraints to enforce the nonlinear $w_{jk} = z_{jk} y_k$.
There are four linear constraints for each $w_{jk}$:

\begin{enumerate}
\item
$y_j - w_{jk} \leq 1 - z_{jk} $
\item
$w_{jk} \leq y_j$
\item
$w_{jk} \leq z_{jk}$
\item
$w_{jk} \geq 0$ 
\end{enumerate}

If $z_{jk} = 1$, then 1 and 2 will ensure that $w_{jk} = y_k$.
If $z_{jk} = 0$, then 3 and 4 will ensure that $w_{jk} = 0$.
It is important to note that this construction wouldn't work if
$y_k > 1$. In our case, this occurs if and only if $z_{jk} = 0$ for all
$j$, which has already been excluded by the $\sum_{j=1}^m z_{jk} \geq 1$
constraint.

Now we are ready to present the mixed integer version
\begin{center}
\bgroup
\def\arraystretch{1.5}
\begin{tabular}{l l l}
maximize   & $a^T w_1 + \cdots + a^T w_K$ \\
subject to 
           & $\begin{cases}
                 1 + z_{jk} \geq x_{hk} + x_{ik} \\
                 z_{jk} \leq x_{hk}             \\
                 z_{jk} \leq x_{ik}             \\
             \end{cases}$
           & $j = 1, ..., m$, $k = 1, ..., K$, $(h,i) = j$ \\
           & $\sum_{j=1}^m z_{jk} \geq 1$ & $k = 1, ..., K$ \\
           & $\sum_{k=1}^K x_{ik} = 1$ & $i = 1, ..., n$ \\
           & $\sum_{i=1}^n x_{ik} \geq 1$ & $k = 1, ..., K$ \\
           & $X \in \{0, 1\}^{n \times K}$ \\
           & $\sum_{j=1}^m w_{jk} = 1$ & $k = 1, ..., K$ \\
           & $\begin{cases}
                y_j - w_{jk} \leq 1 - z_{jk} \\
                w_{jk} \leq y_j \\
                w_{jk} \leq z_{jk} \\
                w_{jk} \geq 0 \\
             \end{cases}$
           & $j = 1, ..., m$, $k = 1, ..., K$ \\
\end{tabular}
\egroup
\end{center}
which can be solved globally by branch-and-bound methods.
Unfortunately the size of our graph is too large for a generic MIP
solver, and the largest graphs we partitioned using this method had
around 400 vertices and 3000 edges, partitioned into 10 components.
This is true even when the binary $\{0, 1\}$ constraint was relaxed
to an interval $[0, 1]$ to create an LP.


\section{An Approximate Solution Using SymBMF}

A faster approximation with fewer variables to this MIP involves
dropping the assignment matrix $X$. The problem becomes

\begin{center}
\bgroup
\def\arraystretch{1.5}
\begin{tabular}{l l l}
maximize   & $\dfrac{a^T z_1}{e_m^T z_1} + \cdots +
              \dfrac{a^T z_K}{e_m^T z_K}$ \\
subject to & $\sum_{k=1}^K z_k = e_m$ \\
           & $e_m^T z_k \geq 1$   & for $k = 1, ..., K$ \\
           & $z_k \in \{0, 1\}^m$ & for $k = 1, ..., K$ \\
\end{tabular}
\egroup
\end{center}

using the \cite{Li:94} transformation we get

\begin{center}
\bgroup
\def\arraystretch{1.5}
\begin{tabular}{l l l}
maximize   & $a^T w_1 + \cdots + a^T w_K$ \\
subject to & $\sum_{k=1}^K z_k = e_m$ \\
           & $e_m^T z_k \geq 1$  & for $k = 1, ..., K$ \\
           & $z_k \in \{0, 1\}^m$ & for $k = 1, ..., K$ \\
           & $\sum_{j=1}^m w_{jk} = 1$ & $k = 1, ..., K$ \\
           & $\begin{cases}
                y_j - w_{jk} \leq 1 - z_{jk} \\
                w_{jk} \leq y_j \\
                w_{jk} \leq z_{jk} \\
                w_{jk} \geq 0 \\
             \end{cases}$
           & $j = 1, ..., m$, $k = 1, ..., K$ \\
\end{tabular}
\egroup
\end{center}

If $K > 1$ and the graph is connected, then not every edge can be
within a parcel; there must be at least one that is between two
different parcels. However, the first constraint
$\sum_{k=1}^K z_k = e_m$ forces the contrary --- every edge must be
assigned to exactly one parcel. Why does this constraint have to be an
equality rather than a $\leq$ as in the first MIP? If it were the latter
the optimal solution to this problem would be trivial: take the $K$
edges with the largest weights in the graph and assign each to a
different parcel. Do not assign any other entries of $z_k$ 1. This
optimal solution is useless for partitioning. This is a consequence of
eliminating the $X$ matrix from the first problem.

If the edges of the graph are over-assigned by $z_k$, how do we recover
the assignment matrix $X$? We propose to create an approximate
partition matrix (\ref{partition_matrix}) $P$ from the $z_k$ so that
\[ P_{ij} = \begin{cases}
  1 & \text{if } \argmax{k} z_{ik} = \argmax{k} z_{jk} \\
  0 & \text{otherwise}
\end{cases}\]
Following this we could use either of the SymBMF methods introduced
in the previous chapter to find a decomposition $P \approx X X^T$.