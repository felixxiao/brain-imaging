
Paragraph 1: Write here what would happen in the ideal world

Paragraph 2: Write here why we cannot reach this ideal easily

Paragraph 3: Write here what we do instead to overcome this problem

\section{fMRI Background}

Functional magnetic resonance imaging (fMRI) is a technique that
measures brain activity. It does this by measuring blood flow and
oxygenation in different parts of the brain. When an area of the brain
is in use, it consumes more oxygen and hence requires more blood.
The fMRI machine detects magnetic signals from the hydrogen atoms of
water molecules in blood. The signal emitted depends on the contents
of the blood, in particular how much oxygen it contains. This detection
method is referred to as blood-oxygen-level dependent, or BOLD.

The fMRI machine detects, for all regions of the brain, the change in
magnetic signal within a given time frame (typically 2 seconds) and
records this as a 3-dimensional image. This image is comprised of units
called voxels, each representing a cube of brain tissue. These 3-D units
are analogous to 2-D pixels that make up computer screens. Each voxel
represent around a million brain cells.

This measurement made repeatedly over all 2 second intervals in the
course of around 10 minutes. The resulting data collected is a time
series of 3-D images.

\section{Parcellation Background}

Functional connectivity -- belief that certain parts of the brain
  work together, even in resting-state [default, baseline] (no stimulus, but still processes going on)

One of the core principles of neuroscience is functional connectivity
-- the idea that 


{\color{blue}
One of the main goals of neuroscience studies is to determine
the resting-state connectivity and functional organization of the brain.
A number of functional networks have been identified associated with
the motor system, language, and memory, in addition to numerous
studies highlighting the resting-state network CITE.
More recently there has been an explosion of interest in applying
network theory (graph logic) methods to the analysis of functional
connectivity data in order to characterize the connectivity between
nodes at the entire brain system level CITE.
Voxel-level network analyses have been developed
that can provide insight into the functional connectivity
of the brain, but these networks are often too complicated to
practically allow further scientific investigation.

Hence, a parcellation-level network is a natural idea where
each parcel represents a group of spatially-adjacent voxels that
have similar functional behaviors. 
Many authors have used atlas-based definitions the Brodmann-based
automatic anatomic labeling atlas (AAL) \cite{tzourio2002automated}
which defines 116 parcels based on the brain anatomy.
While apparently meaningful results have
been obtained with such approaches \cite{hartman2011role, he2009uncovering, liu2008disrupted, lynall2010functional, power2011functional, salvador2005neurophysiological, spoormaker2010development, supekar2008network, tian2011hemisphere, wang2009parcellation}, they are not ideal due to two reasons.
First, while brain anatomy is a reasonable factor in determine
functionally-similar regions, a data-driven parcellation estimated in a
statistical fashion could perform better. Second, anatomically-based
parcellation have a fixed number of parcels. Investigators seeking a
higher-resolution network would need a higher-resolution parcellation,
which is unavailable.

A data-driven parcellation would resolve these problems. 
The bulk of most parcellation methods revolve around recasting our
problem into a clustering problem with spatial constraints.
\cite{biswal2010toward,smith2009correspondence} use Independent
Component Analysis (ICA) to determine regions of voxels whereby the
mean signal from each region are roughly independent with one another.
\cite{chen2008group} extended this idea by using PCA first to determine
the number of clusters, while
\cite{beckmann2005investigations, de2006fmri} extend this idea by
fitting a mixture model to each component found in ICA. Another large
body of works include K-means clustering \cite{flandin2002parcellation,mezer2009cluster,peltier2003detecting,thirion2006dealing}.
Some authors using K-means clustering use the measured fMRI signal as
the covariates while others use the frequency coefficients determined
by the Fourier transform. The authors incorporate spatial distances to
ensure that the resulting parcellations are connected. Hierarchical
clustering is another natural candidate, whereby different authors merge
neighboring voxels according to varying statistics \cite{diez2014novel,bellec2006identification,lu2003region,heller2006cluster}.

The methods developed in our thesis are most similar to spectral
methods. Spectral methods represent another large body of literature to
cluster voxels
\cite{craddock2012whole,van2008normalized,shen2010graph,newman2006modularity,shen2013groupwise,zhang2014robust}.
These methods include spectral clustering or spectral partitioning.
These methods typically form the weight matrix by the distances from
K-nearest neighbors, by Pearson correlation, or regularized linear
regression. As spectral methods tend to split the covariates into two
clusters, these methods are applied recursively onto each cluster.

There is a wide literature of less commonly-used methods.
\cite{alexander2012discovery} recast the problem as a
community-detection problem based on the wavelet frequency.
\cite{baria2011anatomical} bin the frequencies and do local $t$-tests
to determine whether or not two voxels are significantly related.
\cite{cohen2008defining, gordon2014generation, barnes2011parcellation}
recast the problem to be an edge-detection problem using the
developments in computer vision where they input heatmaps formed by
correlation values. The edge-detection algorithm then outputs visually
distinct components based on the images.
\cite{ryali2013parcellation,pohl2007hierarchical} build sophisticated
mixture models for how each voxel value is realized and use an
expectation-maximization (EM) algorithm to uncover the clustering.
Spatial priors can be incorporated into the model to encourage
connected parcellations while multiple initializations can be used as a
heuristic to determine the ``likeliness" of two voxels belonging to the
same parcel. Finally, \cite{blumensath2013spatially} computes the
Kendall's coefficient of concordance at each voxel and its 26 spatial
neighbors and grows clusters based on these values.
}

\section{Overall Strategy}
{\color{red}
Let's move this stuff about correlation to the beginning of the next
chapter about energy statistics. For now, it's good enough to say in
one paragraph that we'll use some nonlinear dependency in our
procedure or something (high level summary)}

To measure dependence, statisticians have traditionally used the Pearson
correlation coefficient, in addition to the rank-based Kendall tau and
Spearman rho. These statistics work well when the underlying
relationship between the two random variables is linear, in the case of
Pearson, or can be linear after a monotonic transformation, in the case
of Kendall and Spearman. Due to their restrictions, these correlation
coefficients will fail to capture many kinds of dependency
relationships. The figure below illustrates several instances of pairs
of random variables whose depencency structure is not detected by the
three correlation coefficients.

\includegraphics[scale = 0.8]{figs/1_nonlinear_depend.png}

Non-linear dependency relationships also exist in the ABIDE 50002 fMRI
data. The scatterplots below show time samples of spatially adjacent
voxels. These instances were found by searching for the maximum
difference in rank of energy distance correlation and the coefficient
of determination, or Pearson squared.

\includegraphics[scale = 0.7]{figs/1_nonlinear_ABIDE_50002.png}

Many studies on functional parcellation (Craddock 2012; Bellec 2006;
Heller 2006) use Pearson's coefficient as the similarity measure between
nearby voxels. Apart from underestimating the important of non-linear
relationships, this method also distinguishes positive, upward-sloping
correlation from negative. As a result in many of the edges between
different parcels, the corresponding voxels would be strongly dependent
with negative correlation.

\section{About the Data}
{\color{blue}

Autism spectrum disorders (ASD) represent a formidable challenge for
psychiatry and neuroscience because of their high prevalence, lifelong
nature, complexity and substantial heterogeneity. Roughly 1\% of
children worldwide are diagnosed with ASD \cite{centers2010autism}.
We approach the parcellation problem by using the Autism Brain Imaging
Data Exchange (ABIDE) -- a consortium aggregating and openly sharing
1112 existing resting-state functional magnetic resonance imaging
(R-fMRI) data sets with corresponding structural MRI and phenotypic
information from 539 individuals with ASDs and 573 age-matched typical
controls (TCs; 7–64 years).

For simplicity in this thesis, we focus on 6 autistic and 6 control
subjects scanned at the University of Pittsburgh School of Medicine.
The autistic subjects included individuals from 7 to 35 years of age,
with a well-characterized Autistic Disorder. The Autism Diagnostic
Interview-Revised \cite{lord1994autism} and the Autism Diagnostic
Observation Schedule-General \cite{lord2000autism}, as well as expert
clinical opinion, were used to diagnose autism. Typical controls were
healthy individuals, with no history of head trauma, birth
complications, seizures, or psychiatric disorder. TC matched
individually to the participants with autism on age (within 1.5 y in
children, 3.5 y in adults), full-scale IQ (within 12 points) and gender.
Individuals with Autistic Disorder were referred from the Center for
Excellence in Autism Research (CEFAR) and the Autism Center of
Excellence (ACE). TC were recruited from previous studies at the LNCD or
by fliers and announcements.
\{If you want to know, I got this information from
\url{http://fcon_1000.projects.nitrc.org/indi/abide/}.
You probably can include this url directly or something.\}

In order to register each subject's brain to a common brain space, we
use the Montreal Neurological Institute's standardized brain
\cite{evans19933d,collins1994automatic}.
The MNI wanted to define a brain that is more representative of the
population. They created a new template that was approximately matched
to the Talairach brain in a two-stage procedure. First, they took 241
normal MRI scans, and manually defined various landmarks, in order to
identify a line very similar to the AC-PC line, and the edges of the
brain. Each brain was scaled to match the landmarks to equivalent
positions on the Talairach atlas. They then took 305 normal MRI scans
(all right handed, 239 M, 66 F, age 23.4 +/- 4.1), and used an
automated 9 parameter linear algorithm to match the brains to the
average of the 241 brains that had been matched to the Talairach atlas.
From this they generated an average of 305 brain scans thus transformed
- the MNI305. The current standard MNI template is the ICBM152, which
is the average of 152 normal MRI scans  that have been matched to the
MNI305 using a 9 parameter affine transform. The International
Consortium for Brain Mapping adopted this, the MNI152, as their
standard template, and this is the template we will be using. We use
the unsymmetrical MRI scan from MNI152 corresponding to voxels
corresponding to cubes with an edge-length of 2 millimeters. This
MNI152 template are $91\times 109 \times 91$ voxels in dimension but
only 228,453 voxels represent the brain (25\% of the total volume).

We preprocessed the raw data from ABIDE using the Configurable Pipeline
for the Analysis of Connectomes (C-PAC) alpha version 0.3.9. C-PAC is
an open-source software pipeline for automated preprocessing and
analysis of resting-state fMRI data. The image preprocessing steps
included slice-timing and motion correction based on the Friston Model,
nuisance signal regression (including 5 CompCorr signals, the
cerebrospinal fluid (CSF), motion and the global, linear, and quadratic
signals) and temporal filtering (0.001-0.08Hz). The derived R-fMRI
measures were normalized to Montreal Neurological Institute (MNI152)
stereostatic space (2mm$^3$ isotropic) with linear regressions and
spatially smoothed (applied FWHM = 6mm).

We then converted the 4-dimensional fMRI data (i.e., a 3-dimensional
image varying with time) in a 2-dimensional matrix whereby each column
represents a different voxel and each row represents a different sample
from a different time. Since C-PAC removed most autocorrelations in the
data, we can reasonably treat each observation (i.e, each row) as drawn
from the same unknown distribution.
}

In this investigation, all parcellation and validation procedures were
conducted on the ABIDE 50002 fMRI data set. This data set contains
233305 voxels and 124 time samples. Spatial information is encoded as a
graph; each voxel is represented by a vertex, and each vertex has up to
6 edges connecting the voxel to its cubically adjacent neighbors. The
weights on the edges are sample energy distance correlations between the
two connected voxels (Szekely 2013).

\section{Notation}
{\color{red}If you don't have any, you can omit this section :p}

\section{Chapter Summaries}
{\color{red}Put the last part of what was currently in your abstract
here.}

