Paragraph 1: Write here what would happen in the ideal world

Paragraph 2: Write here why we cannot reach this ideal easily

Paragraph 3: Write here what we do instead to overcome this problem

\section{fMRI Background}

Functional magnetic resonance imaging (fMRI) is a technique that
measures brain activity by detecting the blood flow and oxygenation
levels in different parts of the brain. When an area of the brain
is in use, it consumes more oxygen and hence requires more blood.
The fMRI machine measures the strength of magnetic signals from the
hydrogen atoms of water molecules in blood. The signal emitted depends
on the contents of the blood, in particular how much oxygen it contains. 
This detection method is referred to as blood-oxygen-level dependent, or
BOLD.

The fMRI machine detects, for all regions of the brain, the change in
magnetic signal within a given time frame (typically 2 seconds) and
records this as a 3-dimensional image. This image is comprised of units
called voxels, each representing a cube of brain tissue. These 3-D units
are analogous to 2-D pixels that make up computer screens. Each voxel
represent around a million brain cells.

This measurement made repeatedly over all 2 second intervals in the
course of around 10 minutes. The resulting data collected is a time
series of 3-D images.

\section{Parcellation Background}

A fundamental goal of neuroscience is to understand the functional
connectivity of the brain. Functional connectivity refers to the set
of statistical dependencies between the activity levels of different
regions of the brain. Numerous studies have highlighted the linkage of
certain brain regions with motor, language, and memory tasks. They
have shown functional connectivity to be heavily dependent on what
actions the subject is performing, the physiological qualities of the
subject, and even the timescale over which the data is collected.

Consequently, neuroscientists developed and focused on the idea of
resting-state functional connectivity. The observation behind this
approach is that the brain exhibits some regular patterns of functional
organization even when the subject is resting and not actively engaged
in a task. (The subject may be daydreaming or mind-wandering.) The goal
of establishing a map of resting-state functional connectivity thus
emerged as a refinement to the goal of understanding the baseline
organization of the brain.

In recent years neuroscientists have increasingly viewed the analysis
of functional connectivity through the lens of graph theory. The brain
was modeled as a graph with vertices representing fMRI voxels and edges
representing connections between voxels with weights defined by some
measure of statistical dependency.

The problem with this framework is that the sheer size, complexity, and
detail of the voxel graphs, coupled with the variation in brain structure
between individuals, made it too difficult to draw scientific
conclusions from them. A coarser map was needed, one that ideally would
retain the most useful information in the voxel-level graphs.

Hence the idea of parcellation emerged. Brain parcellation roughly means
the task of grouping spatially-adjacent voxels so as to maximize the
statistical dependence of signals arising from voxels in the same group.
Early attempts at parcellation were focused on brain anatomy. For
instance the popular Automatic Anatomic Labeling atlas (AAL)
\cite{tzourio2002automated} defines 116 parcels based on the similarity
of brain tissue and other structural information. Many findings based on
this parcellation have been published \cite{hartman2011role,
he2009uncovering, liu2008disrupted, lynall2010functional,
power2011functional, salvador2005neurophysiological,
spoormaker2010development, supekar2008network, tian2011hemisphere,
wang2009parcellation}. However, the AAL parcellation suffers two major
shortfalls. First, it is not based on statistical dependence. Second,
it cannot adapt to accomodate researchers who want more than 116 parcels
and a higher-resolution network. The answer to these issues lies in
data-driven parcellations based on functional connectivity.

The bulk of data-driven parcellations in the literature are based on
spatially-constrained clustering methods that relate to some approximate
measure of statistical dependency such as Pearson correlation.
To list a few, \cite{biswal2010toward, smith2009correspondence} and
\cite{chen2008group} use Independent Component Analysis and Principal
Components Analysis. \cite{beckmann2005investigations, de2006fmri} and 
\cite{ryali2013parcellation} builds a mixture model
for how the voxel signals is generated and cluster using an
Expectation-Maximization algorithm.

Another large body of work uses K-means clustering
\cite{flandin2002parcellation, mezer2009cluster,peltier2003detecting,
thirion2006dealing} with some spatial contraints imposed to ensure
connectivity. Hierarchical clustering methods used in
\cite{diez2014novel, bellec2006identification, lu2003region,
heller2006cluster, blumensath2013spatially, pohl2007hierarchical}
have the advantage of naturally producing connected parcels.

Spectral methods make up another large body of literature on voxel
clustering \cite{craddock2012whole, van2008normalized, shen2010graph,
newman2006modularity, shen2013groupwise, zhang2014robust}.
These methods, which use the eigenvectors of the graph Laplacian matrix,
are characterized by the choice of edge weights (either Pearson's
correlation or Gaussian kernel) and the choice of cut objective
(usually normalized cut) and can be either recursive or multiway.
For details on these methods, see Chapter 5.

There is a wide literature of less commonly-used methods.
\cite{alexander2012discovery} uses a method based on normalized mutual
information.
\cite{cohen2008defining, gordon2014generation, barnes2011parcellation}
adapt edge-detection algorithms for planar graphs from computer vision.

\section{Overall Strategy}
We continue in the tradition established by the aforementioned authors
of framing the problem of brain parcellation as a problem of graph
partitioning. Our contribution to this field is four-part:

\begin{enumerate}
\item
We introduce the usage of a fairly new statistic for measuring
dependency: distance correlation. Unlike traditional statistics such
as Pearson's correlation, distance correlation has the special property
that it is 0 if and only if the two random variables are independent.
This brings brain parcellation back to framework of mapping functional
connectivity defined as statistical dependence of voxel actvity.

\item
We introduce the novel graph partitioning objective of Maximize Average
Within-Edge (MAWE) and justify its usage in conjunction with distance
correlation. We also develop metrics for evaluating the smoothness and
size balance of parcellations.

\item
We propose a number of methods, some original and some borrowed, for
approximately solving the graph partitioning problem of MAWE with
constraints on parcel smoothness and size.

\item
We carry out these methods on fMRI data collected from patients with
and without autism spectrum disorder.
\end{enumerate}

\section{About the Data}

Autism spectrum disorders (ASD) represent a formidable challenge for
psychiatry and neuroscience because of their high prevalence, lifelong
nature, complexity and substantial heterogeneity. Roughly 1\% of
children worldwide are diagnosed with ASD \cite{centers2010autism}.
We obtained data for our parcellation problem from the Autism Brain
Imaging Data Exchange (ABIDE) -- a consortium that has collected
1112 resting-state fMRI data sets collected from 539 individuals
with ASDs and 573 age-matched typical controls.

For simplicity in this thesis, we focus on 6 autistic and 6 control
subjects scanned at the University of Pittsburgh School of Medicine.
The autistic subjects included individuals from 7 to 35 years of age,
with a well-characterized Autistic Disorder. Patients were diagnosed
with the Autism Diagnostic Interview-Revised \cite{lord1994autism} and
the Autism Diagnostic Observation Schedule-General \cite{lord2000autism},
under expert clinical opinion. Typical controls were healthy
individuals, with no history of head trauma, birth complications,
seizures, or psychiatric disorder. The controls were also matched by
age, full-scale IQ, and gender with patients in the ASD group.
For more information, refer to
\url{http://fcon_1000.projects.nitrc.org/indi/abide/}.

In order to register each subject's brain to a common brain space, we
use the Montreal Neurological Institute's (MNI) standardized brain
\cite{evans19933d, collins1994automatic}.
The MNI standardized brain template MNI152 was taken from the average
of 152 normal MRI scans. We use the unsymmetrical MRI scan with voxels
corresponding to cubes with an edge-length of 2 millimeters. The
dimensions of this MNI152 template are $91 \times 109 \times 91$ voxels
with 228,453 voxels representing space occupied by the brain.

We preprocessed the raw data from ABIDE using the Configurable Pipeline
for the Analysis of Connectomes (C-PAC) alpha version 0.3.9. C-PAC is
an open-source software pipeline for automated preprocessing and
analysis of resting-state fMRI data. The image preprocessing steps
included slice-timing and motion correction, nuisance signal regression
and temporal filtering. The derived resting-state fMRI data were
normalized to the MNI152 template.

Following C-PAC preprocessing, we converted the 4-dimensional fMRI data
(a 3-dimensional image varying with time) to a 2-dimensional matrix
with each column representing a different voxel and each row a time
sample. Since C-PAC removed most autocorrelations in the data, we can
reasonably treat each row observation as independently from the same
unknown distribution.

\section{Chapter Summaries}
Chapter 2 presents the theory of distance correlation, which
belongs to the larger family of energy statistics. We argue for using
distance correlation as a better alternative to Pearson's in problems
of dependence-based clustering.

Chapter 3 defines the set of valid parcellations and introduces
validation criteria for measuring how well a parcellation captures
functional connectivity. The validation criteria are based on distance
correlation. In particular, we introduce the objective of Maximize
Average Within-Edge (MAWE) that serves as the primary measure of
statistical dependency within parcels throughout this thesis.
Additionally, the chapter presents metrics for quantifying size balance, 
smoothness, and similarity of two parcellations.

Chapter 4 details the first class of parcellation methods that are
based on graph-growing heuristics. In particular, we introduce a new
data structure called the Contractible Graph and its associated
algorithm Edge-Contract. We illustrate the efficient implementation of
Edge-Contract and show that it can be flexibly extended to encourage
parcellations that are smooth and size-balanced.

Chapter 5 explores the well-established spectral methods for graph
partitioning, beginning with an exposition on the early bipartitioning
method and then moving on to its more sophisticated multiway version
that minimizes the ratio-cut of a partitioning and requires K-means.

Chapter 6 begins with a very recently developed clustering method based
on a symmetric factorization of the adjacency matrix into nonnegative
matrices and illustrates the connection between this and spectral
multiway partitioning. The second half discusses the case of 0-1
symmetric matrix factorization and presents two original approaches.

Chapter 7 presents two novel formulations of a connectivity-relaxed
version of MAWE as a generalized 0-1 fractional program which has an
equivalent mixed integer linear program. In particular, we derive both a
globally optimal formulation and an more efficient approximate one.

Chapter 8 compares all of the aforementioned parcellation methods and
presents the parcellation results of some of them on the ABIDE fMRI data.
