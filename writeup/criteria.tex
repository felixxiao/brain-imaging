% 1. in between-score and boundary-score, split summation indexation
%    into 2 lines
%#######################################################################

In chapter one we discussed our graphical approach to the brain
parcellation problem. We construct a weighted undirected graph where
each vertex is a voxel. The graph reflects the spatial position of the
voxels; it connects each vertex to the vertices representing the voxel's
six cubically adjacent neighbors. The weights on these edges are sample
energy distance correlation statistics between the adjacent voxels in
the time series and they measure statistical dependence between the
voxels. Let $G(V, E)$ denote this graph, its vertices, and its edges.

In this context, a valid $k$-fold partition $\mathcal{P}_k$ of the
graph $G$ is a collection of vertex subsets $(V_1, ..., V_k)$
satisfying the following:

\begin{enumerate}[1.]
\item
$V_i \neq \emptyset$ for all $V_i \in \mathcal{P}_k$

\item
$\bigcup\limits_{i=1}^k V_i = V$

\item
$V_i \cap V_j = \emptyset$ for all $V_i, V_j \in \mathcal{P}_k$

\item
$V_i$ is connected (i.e. for every two vertices in $V_i$, there is a
path between them) for all $V_i \in \mathcal{P}_k$
\end{enumerate}

In this chapter we will suggest various criteria for measuring the
goodness-of-fit of partitions. We will argue why these criteria are
sensible from a neuroscience perspective.

\section{Within-Parcel Similarity}

Voxels in the same parcel are ideally highly dependent on one another in
the time series of fMRI data. As discussed in the previous chapter,
distance correlation is a good measure of dependence. The distance
correlation between two random vectors equals zero if and only if the
two random vectors are independent, which is not true of correlation
statistics such as Pearson's.

Let $\mathcal{R}(x,y)$ denote the distance correlation between two
voxels $x$ and $y$.
Let $V$ and $W$ be any parcels. We will use $E_V$ to denote the set of edges with one end in $V$ and one end not in $V$, and $E_{V,W}$ the set
of edges with one end in $V$ and one in $W$.

\begin{definition}[Within-Score] \label{within-score}
\[ \frac{1}{k} \sum_{V \in \mathcal{P}_k}
   \frac{1}{|V|^2} \sum_{x,y \in V} \mathcal{R}(x,y)
\]
\end{definition}

The Within-Score is non-spatial; it considers all pairs of voxels
equally regardless of whether they are adjacent. As a result, it is a
good measure of how much the voxels within each parcel are dependent on
each other as a set. The downside of this criterion is that it is very
expensive to compute. With over 300,000 voxels in an fMRI data set we
would potentially have to compute tens of billions of distance
correlation statistics if the number of parcels is small.

There are two solutions to this. One is to subsample: for every parcel,
compute the distance correlation matrix for a small subset of the voxels
in the parcel and construct a confidence interval around an estimate of
the Within-Score. Another solution is to reduce the image resolution:
merge multiple adjacent voxels into a single voxel by averaging the time
series and then compute the Within-Score.

An alternative and far less expensive criterion that measures within-
parcel similarity works by counting distance correlations between
adjacent pairs of voxels.

\begin{definition}[Adjacent-Score] \label{adjacent-score}
\[ \frac{1}{k} \sum_{V \in \mathcal{P}_k}
   \frac{1}{|E_{V,V}|} \sum_{(x,y) \in E_{V,V}} \mathcal{R}(x,y)
\]
\end{definition}

Rather than treat parcels as sets with no spatial information, the
Adjacent-Score does the opposite by only considering the pairwise
dependency of adjacent voxels.

Other possibilities that we did not explore are considering all pairs
of voxels up to some maximum spatial distance from each other and
performing a weighted averaging of sample pairwise distance
correlations, with weights that depend on spatial distance.

\section{Between-Parcel Dissimilarity}

To evaluate parcellation quality, it is also useful to measure how
dependent voxels belonging to different parcels are on each other. To
this end we define two criterion similar to the Within-Parcel criterion;
a non-spatial metric called the Between-Score and a spatial metric
called the Boundary-Score.

\begin{definition}[Between-Score] \label{between-score}
\[ \frac{1}{\binom{k}{2}} \sum_{V, W \in \mathcal{P}_k, V \neq W}
   \frac{1}{|V||W|} \sum_{x \in V, y \in W} \mathcal{R}(x,y)
\]
\end{definition}

\begin{definition}[Boundary-Score] \label{boundary-score}
\[ \frac{1}{\binom{k}{2}} \sum_{V, W \in \mathcal{P}_k, V \neq W}
   \frac{1}{|V||W|} \sum_{(x,y) \in E_{V,W}} \mathcal{R}(x,y)
\]
\end{definition}

For computational reasons, parcellation algorithms will be evaluated
primarily using the spatial Adjacency- and Boundary- scores.

\section{Graph Cuts}

Closely related to the Boundary-Score is the notion of a graph cut.

\section{Balance and Compactness}

\begin{definition}[Balance-Score] \label{balance-score}
\[ 1 - \frac{1}{\binom{k}{2}} \frac{1}{\max_{V \in P_k} |V|}
   \sum_{V,W \in P_k : V \neq W} \big| |V| - |W| \big| \]
\end{definition}

The Balance-Score ranges from 1 (all equally sized parcels) to
0 (two parcels with one of size zero).

\begin{definition}[Compactness-Score] \label{compactness-score}
\[ \frac{1}{k} \sum_{V \in \mathcal{P}_k} \frac{|E_V|^\frac{3}{2}}{|V|}
\]
\end{definition}

The $\frac{3}{2}$ power makes the ratio invariant on parcel size.
For instance, a $n \times n \times n$ cube of vertices would have
a compactness of $6^{\frac{3}{2}}$.

\section{Comparing Multiple Parcellations: Similarity}
