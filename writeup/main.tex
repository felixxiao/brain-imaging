\documentclass{report}

\title{Approaches to Brain Parcellation using Energy Statistics and Graph Partitioning}
\author{Felix Xiao}

\usepackage[a5paper, margin = 1cm]{geometry}
\usepackage{amsmath, amsfonts, graphicx, enumerate, amssymb, algpseudocode, algorithm, csvsimple}

\newtheorem{definition}{Definition}[section]
\newtheorem{prop}[definition]{Proposition}
\newtheorem{lemma}[definition]{Lemma}
\newtheorem{corollary}[definition]{Corollary}
\newtheorem{theorem}[definition]{Theorem}

\newcommand{\R}{\mathbb{R}}
\newcommand{\Expect}{{\rm I\kern-.3em E}}
\newcommand{\Var}{\mathrm{Var}}
\newcommand{\Cov}{\mathrm{Cov}}
\newcommand{\indep}{\rotatebox[origin=c]{90}{$\models$}}
\newcommand{\argmin}[1]{\underset{#1}{\mathrm{arg min}}\;}
\newcommand{\argmax}[1]{\underset{#1}{\mathrm{arg max}}\;}
\newcommand{\norm}[1]{\left\lVert#1\right\rVert}

\DeclareMathOperator{\Tr}{Tr}
\let\emptyset\varnothing

\begin{document}

\maketitle
\tableofcontents

\begin{abstract}
We formulate the task of brain parcellation (dividing the
brain into functionally homogenous regions) as a graph
partitioning problem. We devise new model-free criteria
for validating parcellations based on statistical
dependency between parcels using distance correlation.
Based off our criteria we pose a new graph cut-type
problem called Max Average Within Edge, wherein the
objective is to maximize, for each component, the summed
weight of all edges with both endpoints in the component
divided by the number of edges within the component.

We propose a family of heuristic algorithms for attaining
good solutions for MAWE.
\end{abstract}

%\chapter{Functional MRI Data and Brain Parcellation}
%%#######################################################################

    [Need to find somewhere to put this]
    I conducted all parcellation and validation procedures on the ABIDE
    50002 fMRI data set provided by Kevin Lin. This data set contains 233305
    voxels and 124 time samples. Spatial information is encoded as a graph;
    each voxel is represented by a vertex, and each vertex has up to 6 edges
    connecting the voxel to its cubically adjacent neighbors. The weights on
    the edges are sample energy distance correlations between the two
    connected voxels (Szekely 2013).

Functional parcellation of the human brain can be defined as the problem
of partitioning the voxels into $k$ disjoint connected components with
the goal that the voxels within each component are, in a rough sense,
\"similar\" to each other and voxels in different components are less
\"similar\". Such similarity has been defined in a multitude of ways in
the literature [see lit review section ...]. For this project thus far I
have taken similarity between voxels to mean statistical dependence.

To measure dependence, statisticians have traditionally used the Pearson
correlation coefficient, in addition to the rank-based Kendall tau and
Spearman rho. These statistics work well when the underlying
relationship between the two random variables is linear, in the case of
Pearson, or can be linear after a monotonic transformation, in the case
of Kendall and Spearman. Due to their restrictions, these correlation
coefficients will fail to capture many kinds of dependency
relationships. The figure below illustrates several instances of pairs
of random variables whose depencency structure is not detected by the
three correlation coefficients.

[ insert figure here ]

Non-linear dependency relationships also exist in the ABIDE 50002 fMRI
data. The scatterplots below show time samples of spatially adjacent
voxels. These instances were found by searching for the maximum
difference in rank of energy distance correlation and the coefficient
of determination, or Pearson squared.

\includegraphics[scale = 0.7]{1_nonlinear_ABIDE_50002.png}

Many studies on functional parcellation (Craddock 2012; Bellec 2006;
Heller 2006) use Pearson's coefficient as the similarity measure between
nearby voxels. Apart from underestimating the important of non-linear
relationships, this method also distinguishes positive, upward-sloping
correlation from negative [fact check needed here]. As a result in many
of the edges between different parcels, the corresponding voxels would
be strongly dependent with negative correlation [fact check needed here].


%\chapter{Energy Statistics}
%%#######################################################################

\section{Energy Covariance}

For some positive weight function
$w : \R^p \times \R^q \mapsto [0, \infty)$ define the norm
$\|\cdot\|_w : \{\gamma : \R^p \times \R^q \mapsto \mathbb{C}\}
               \mapsto [0, \infty)$ as
$$ \| \gamma \|_w^2 = \int_{\R^{p+q}} | \gamma(s,t) |^2 w(s,t) ds dt $$

\begin{definition}
(Distance covariance). Let $X$ and $Y$ be two $d$-dimensional random
vectors with $\Expect\|X\| + \Expect\|Y\| < \infty$. Their distance
covariance is
\begin{align*}
\mathcal{V}^2 (X,Y)
&= \| \varphi_{X,Y}(s,t) - \varphi_X(s)\varphi_Y(t) \|_w^2 \\
&= \int_{\R^{p+q}} \dfrac{|\varphi_{X,Y}(s,t) - \varphi_X(s)\varphi_Y(t)|^2}
                         {\|s\|^{1+p} \|t\|^{1+q}} ds dt
\end{align*}
where $w(s,t) = \dfrac{1}{\|s\|^{1+p} \|t\|^{1+q}}$.
\end{definition}

It is clear that $\mathcal{V}^2(X,Y) = 0 \iff X \indep Y$.

\begin{prop}
\begin{align*}
\mathcal{V}^2 (X,Y)
&= \Expect[\|X - X'\| \|Y - Y'\|]
 + \Expect[\|X - X'\|] \Expect[\|Y - Y'\|]
 - 2 \Expect[ \|X - X'\| \|Y - Y''\| ] \\
&= \Cov( \|X - X'\|, \|Y - Y'\| ) - 2 \Cov (\|X - X'\|, \|Y - Y''\| ) 
\end{align*}
\end{prop}
\textit{Proof.}

\begin{definition}
(Distance variance).
$$ \mathcal{V}^2 (X) = \mathcal{V}^2 (X,X) $$
\end{definition}

\begin{definition}
(Distance correlation).
$$ \mathcal{R}^2 (X,Y) = \dfrac{\mathcal{V}^2(X,Y)}
                               {\mathcal{V}(X) \mathcal{V}(Y)} $$
\end{definition}

For iid sample realizations $\{(X_i, Y_i)\}_1^n$, let
$\widehat{\varphi_X} (t) = \frac{1}{n} \sum_{i=1}^n e^{i t^T X_i}$ be
the empirical characteristic function for $X$ and likewise for $Y$. An
estimate of $\mathcal{V}^2(X,Y)$ replaces the unknown characteristic
functions with the empirical characteristic functions.

\begin{prop}
$$ \widehat{\mathcal{V}}^2 (X,Y) \equiv
\| \widehat{\varphi_{X,Y}}(s,t) - \widehat{\varphi_X}(s) \widehat{\varphi_Y}(t) \|_w^2 = S_1 + S_2 - 2 S_3 $$
where $w(s,t)$ as above and
\begin{align*}
S_1 &= \frac{1}{n^2} \sum_{k=1}^n \sum_{l=1}^n \|X_k - X_l\| \|Y_k - Y_l\| \\
S_2 &= \left( \frac{1}{n^2} \sum_{k=1}^n \sum_{l=1}^n \|X_k - X_l\| \right) \frac{1}{n^2} \sum_{k=1}^n \sum_{l=1}^n \|Y_k - Y_l\| \\
S_3 &= \frac{1}{n^3} \sum_{k=1}^n \sum_{l=1}^n \sum_{m=1}^n \|X_k - X_l\| \|Y_k - Y_m\|
\end{align*}
Alternatively, we can let $A, B \in \R^{n \times n}$ such that
$A_{kl} = \|X_k - X_l\|$ and $B_{kl} = \|Y_k - Y_l\|$ ($A$ and $B$ are
symmetric elementwise nonnegative). Let
$\overline{X} = \frac{1}{n^2} \sum_{k,l = 1}^n X_{kl}$. Then
$$ \hat{\mathcal{V}}^2 (X,Y)
 = \overline{A \circ B} + \overline{A} \cdot \overline{B} - \frac{2}{n}
   \overline{(A B)} $$
where $\circ$ means element-wise multiplication.
\end{prop}

Estimates of distance variance and distance correlation are defined analogously.

\begin{prop}
$$ \mathcal{V}(v_1 + a_1 Q_1 X, v_2 + a_2 Q_2 Y)
 = \sqrt{|a_1 a_2|} \mathcal{V}(X,Y) $$
\end{prop}

\begin{definition}
($\alpha$-distance covariance). For $0 < \alpha < 2$
$$ \mathcal{V}_\alpha^2 (X,Y)
 = \frac{1}{C(p,\alpha) C(q,\alpha)}
   \int_{\R^{p+q}} \frac{|\varphi_{X,Y}(s,t) - \varphi_X(s) \varphi_Y(t)|^2}
                        {\|s\|^{\alpha + p} \|t\|^{\alpha + q}} ds dt $$
\end{definition}

\begin{prop}
If $\Expect[\|X\|^\alpha] + \Expect[\|Y\|^\alpha] < \infty$ then
$$ \mathcal{V}_\alpha^2 (X,Y) = \Expect[ \|X - X'\|^\alpha \|Y - Y'\|^\alpha ] + \Expect \|X - X'\|^\alpha \Expect \|Y - Y'\|^\alpha - 2 \Expect[ \|X - X'\|^\alpha \|Y - Y''\|^\alpha ]$$
\end{prop}

\begin{corollary}
For $\alpha = 2$, $p = q = 1$, the distance correlation is the absolute value of Pearson's correlation coefficient.
\end{corollary}


\chapter{Criteria for Evaluating Parcellations}
In Chapter 1 we discussed our graphical approach to the brain
parcellation problem. We construct a weighted undirected graph where
each vertex corresponds with a voxel. The graph reflects the spatial
position of the voxels; it connects each vertex to the vertices
representing the voxel's six cubically adjacent neighbors.

The weights on these edges are sample distance correlation statistics
$\mathcal{R}_n(X,Y)$ between the adjacent voxels $X$ and $Y$ in the
time series of fMRI data. The properties of $\mathcal{R}$ are presented
more thoroughly in the preceding Chapter. The most relevant one is that
$0 \leq \mathcal{R}_n(X,Y) \leq 1$, with higher distance correlation
indicating a greater degree of statistical dependency.

Let $G(V, E)$ denote the voxel graph, its vertices, and its edges.
A valid $k$-fold partition $\mathcal{P}_k$ of the graph $G$ is a
collection of vertex subsets $(V_1, ..., V_k)$ satisfying the following:

\begin{enumerate}[1.]
\item
$V_i \neq \emptyset$ for all $V_i \in \mathcal{P}_k$

\item
$\bigcup\limits_{i=1}^k V_i = V$

\item
$V_i \cap V_j = \emptyset$ for all $V_i, V_j \in \mathcal{P}_k$

\item
$V_i$ is connected (i.e. for every two vertices in $V_i$, there is a
path between them) for all $V_i \in \mathcal{P}_k$
\end{enumerate}

In this chapter we will suggest various criteria for measuring the
goodness of parcellations and discuss their statistical and
computational advantages and drawbacks. Our notation will be as follows:
Let $\mathcal{R}_n(x,y)$ denote the sample distance correlation between
two voxels $x$ and $y$. For any two parcels $V, W \in \mathcal{P}_k$ we
will use $E_V$ to denote the set of edges with one endpoint in $V$ and
one endpoint not in $V$, and $E_{V,W}$ the set of edges with one
endpoint in $V$ and one in $W$.

\section{Within-Parcel Dependency}

Voxels in the same parcel are ideally highly dependent on one another in
the time series of fMRI data. To measure the degree of statistical
dependence within a parcel, we begin with the idea of computing the
sample distance correlation between \textit{all} pairs of voxels in the
same parcel. We'll call this criterion the \textit{Within-Score}.
A good parcellation will have a large Within-Score.

\begin{definition}[Within-Score] \label{within-score}
\[ \frac{1}{k} \sum_{V \in \mathcal{P}_k}
   \frac{1}{|V|^2} \sum_{x,y \in V} \mathcal{R}(x,y)
\]
\end{definition}

The Within-Score is non-spatial; it considers all pairs of voxels
equally regardless of whether they are adjacent. Consequently, it is a
good measure of how much the voxels within each parcel are dependent on
each other as a set. The disadvantage of this criterion is that it is
very expensive to compute. With over 200,000 voxels in an fMRI data set
we would potentially have to compute tens of billions of distance
correlation statistics, each of which takes time proportional to the
number of samples squared.

An alternative and far less expensive criterion that measures within-
parcel similarity works by counting distance correlations between
adjacent pairs of voxels.

\begin{definition}[Adjacent-Score] \label{adjacent-score}
\[ \frac{1}{k} \sum_{V \in \mathcal{P}_k}
   \frac{1}{|E_{V,V}|} \sum_{(x,y) \in E_{V,V}} \mathcal{R}(x,y)
\]
\end{definition}

Rather than treat parcels as sets with no spatial information, the
Adjacent-Score does the opposite by only considering the pairwise
dependency of adjacent voxels. For sparse graphs such as ours, the
number of distance correlation computations is proportional to the
number of vertices. In our cubically adjacent voxel graph, it is bounded
above by $6 |V|$. Both the Adjacent-Score and Within-Score are
between 0 and 1.

We define the Maximize Average Within-Edge (MAWE) for $k$ partitions
problem as the problem of finding a valid $k$-fold partition
$\mathcal{P}_k$ of $V$ so as to maximize the Adjacent-Score
(\ref{adjacent-score}). MAWE will serve as the general guideline for our
parcellation algorithms in later chapters. Hence we adopt the
Adjacent-Score as our primary metric for evaluating the goodness of
parcellations for capturing functional connectivity.

\section{Between-Parcel Dependency}

Another way of viewing parcellation quality is to look at how
dependent voxels belonging to different parcels are on each other. To
this end we define two criterion similar to the Within-Parcel criterion;
a non-spatial metric called the Between-Score and its spatial metric
the Boundary-Score. Contrary to the within parcel criteria presented
above, the goal now becomes to \textit{minimize} the criteria measuring
between-parcel dependency.

\begin{definition}[Between-Score] \label{between-score}
\[ \frac{1}{\binom{k}{2}} \sum_{V, W \in \mathcal{P}_k, V \neq W}
   \frac{1}{|V||W|} \sum_{x \in V, y \in W} \mathcal{R}(x,y)
\]
\end{definition}

\begin{definition}[Boundary-Score] \label{boundary-score}
\[ \frac{1}{\binom{k}{2}} \sum_{V, W \in \mathcal{P}_k, V \neq W}
   \frac{1}{|E_{V,W}|} \sum_{(x,y) \in E_{V,W}} \mathcal{R}(x,y)
\]
\end{definition}

Generally both of these quantities are more expensive to compute than
their Within-Parcel counterparts. Boundary-Score is easy enough to
compute for validation purposes, but does not convey much additional
information beyond what the Adjacency-Score does, in the sense that
the edges used in the computation of Adjacency-Score are the complement
of the edges used in the Boundary-Score.

The ability of distance correlation to generalize to pairs of random
vectors of arbitrary dimension gives us another way of computing
the dependency between two parcels. The Multivariate Between-Score
defined below treats parcels as random vectors and computes the distance
correlation at the parcel level rather than voxel level. The result is
a measure of non-spatial between-parcel similarity that is also
computationally feasible. For this reason we will use Multivariate
Between-Score as our primary measure of parcel dissimilarity.
Between-Score, Boundary-Score, and Multivariate Between-Score all
lie between 0 and 1.

\begin{definition}[Multivariate Between-Score]
\label{multi-between-score}
\[ \frac{1}{\binom{k}{2}} \sum_{V, W \in \mathcal{P}_k, V \neq W}
   \mathcal{R}(V, W)
\]
\end{definition}

Closely related to the Boundary-Score is the notion of a graph cut from
computer science. A \textit{cut} is the set of edges with endpoints
in different parcels. The \textit{cut weight} is the sum of weights of
all edges in the cut set and can be expressed as
\[ \frac{1}{2} \sum_{V \in \mathcal{P}_k}
   \sum_{x,y \in E_V} \mathcal{R}(x,y) \]
The \textit{ratio cut} defined below is a weighted version of the cut
weight
\[ \frac{1}{2} \sum_{V \in \mathcal{P}_k} \frac{1}{|V|}
   \sum_{x,y \in E_V} \mathcal{R}(x,y)
\]
Although we do not use these two criteria directly in evaluating
parcellations, we include them here because of their importance in
the graph partitioning literature. In particular, the ratio cut has
a close connection with spectral partitioning methods explored in
Chapter 5.

\section{Balance and Jaggedness}

The previous criteria are concerned solely with measuring functional
connectivity within and between parcels, without regard for the spatial
shape of the parcels.
Both anatomical and functional parcellations in the literature exhibit
some degree of parcel shape regularity. The number of voxels in each
parcel does not vary too much, and the surface of parcels tend to be
smooth. We quantify these two attributes with the following criteria:

\begin{definition}[Balance] \label{balance}
\[ \frac{1}{k} \frac{1}{\underset{V \in \mathcal{P}_k}{\max} |V|}
   \sum_{V \in \mathcal{P}_k} |V| \]
\end{definition}

The Balance-Score has a maximum value of 1 which occurs if and only if
all parcels are equally sized. It is bounded asymptotically below by 0
and approaches this number as $k$ increases and there is one huge
parcel and $k - 1$ miniscule ones. In the Automatic Anatomical
Parcellation, the Balance-Score is around $0.3$. Our parcellations will
aim for around this number.

\begin{definition}[Jaggedness] \label{jaggedness}
\[ \frac{1}{k} \sum_{V \in \mathcal{P}_k} \frac{|E_V|^\frac{3}{2}}{|V|}
\]
\end{definition}

The Jaggedness criterion is a normalized graphical version of the mean
surface area to volume ratio of all parcels. The surface area here is
the number (not weight) of edges with only one endpoint in a parcel and
the volume is the number of vertices in the parcel.

The Jaggedness criterion is \textit{normalized} in the sense that the
$\frac{3}{2}$ power in the numerator makes the ``dimensionality'' of the
surface area (which is 2-D in the 3-D space) equal to dimensionality of
the volume (which is 3-D). This has the benefit of making the
surface-area to volume ratio not depend on the size of the parcel.
For instance, a $n \times n \times n$ cube of vertices would have
a jaggedness of $6^{\frac{3}{2}}$ which does not depend on $n$.

\section{Comparing Multiple Parcellations}

Since we'll be running our parcellation methods on multiple brains we
require a criterion that compares the similarity of two different
parcellations. We will be using the Adjusted Rand Index, which is a
common measure used in the clustering literature.

\begin{definition}[Adjusted Rand Index] \label{ari}
Let $\mathcal{P}_k = \{V_1, ..., V_k\}$ and
    $\mathcal{Q}_l = \{W_1, ..., W_l\}$ be two partitionings of $V$.
Let the overlap of $V_i$ and $W_j$ be denoted 
\[ n_{ij} = | V_i \cap W_j | \]
and let $a_i = \sum_{j=1}^l n_{ij}$ be the row sums and
        $b_j = \sum_{i=1}^k n_{ij}$ be the column sums of $[n_{ij}]$.
The Adjusted Rand Index (ARI) is defined as
\[ \frac{\sum_{i,j} \binom{n_{ij}}{2} -
         \frac{S}{\binom{n}{2}} 
        }
        {\frac{S}{2} - \frac{S}{\binom{n}{2}}} \]
where $S = \sum_i \binom{a_i}{2} + \sum_j \binom{b_j}{2}$
\end{definition}

The advantage of ARI is that the number of parcels in the two
parcellations compared do not have to be equal. The ARI has a value of
1 if and only if the two parcellations assign the same voxels to each
parcel. ARI values close to 0 indicate that parcellations are
independent of each other for any number of parcels, which would occur
if one of the two parcellations were randomly generated. It is possible
to get negative ARI values.


\chapter{Local Search and Graph Growing Heuristics}
% 1. in union-find subsection make function names in text a different
%    font
% 2. include parameters of size-constrained add-edge on randomized graph
%    in table
% 3. more analysis of SC AE results
%#######################################################################

We introduce several algorithms for generating brain parcellations. The
algorithms in this chapter are all local search heuristics; they begin
with $n$ unconnected vertices and iteratively join adjacent ones into
components until some stopping criterion is met.

For each algorithm, the resulting parcellation is presented, discussed,
and evaluated according to the criteria introduced in the previous
chapter.

\section{Unconstrained Add-Edge}

The first and simplest algorithm starts with an empty graph of $n$
vertices and sequentially adds edges between adjacent voxels in order of
highest sample distance correlation, until the graph has some
prespecified number of connected components $k$.

We will refer to this algorithm as ``Unconstrained Add-Edge''. A naive
implementation of would re-compute the number of connected components in
the graph (using linear-time bread-first or depth-first search) after
each addition of an edge, resulting in a costly $O(EN)$ time complexity.
A more efficient implementation takes advantage of the fact that each
addition of an edge decreases the number of components in the graph by
at most 1. Hence the algorithm needs only to compute the number of
connected components after adding $c - k$ edges, where $c$ is the
current number of connected components of the graph, beginning at $n$.

Another implementation uses a binary search-type strategy and is
$O((n + E) \log E)$. The idea is to ``search'' for the last edge to add
to the graph by maintaining a range of possible last edges. In each
iteration, the algorithm would add to the graph edges 1 to the midpoint
of this range, compute the number of connected components, and adjust
the range based on whether the number of components is higher or lower
than the target $K$.

The Unconstrained Add-Edge algorithm produces severely imbalanced
parcellations. In the 100-component graph, there was one component
containing over 99.9\% of all the vertices in the graph. This leads
to a modification that prevents some edges from being added when a
size constraint is violated.

\section{Size-Constrained Add-Edge}

The Size-Constrained Add-Edge algorithm works in a similar manner to the
unconstrained version, adding edges to the graph in decreasing order of
distance correlation. The Size-Constrained version differs by applying
a filter to each edge considered, adding the edge only if at least one
of the two following conditions are met:

\begin{enumerate}[1.]
\item
At least one of the two components bridge by the edge is of size less
than some prespecified parameter $s_{\min}$.

\item
The union of the two components is of size $\leq s_{\max}$.
\end{enumerate}

The restriction on adding new edges was not successful in creating
balanced partitions. For sake of completeness, we documented our
implementation.

The naive implementation must use BFS/DFS in each iteration to compute
the sizes of the two components to be connected by an edge, and hence
must have time complexity $O(EN)$. Fortunately, there is a way to
sublinearly update information on the components of the graph, using
the union-find data structure.

\subsection{Union-Find}

The core Union-Find data structure begins with an empty graph of $N$
vertices and supports two operations. union(i, j) adds an edge between
vertices $i$ and $j$. root(i) returns an identifier for the component
to which vertex $i$ belongs. All vertices in the same component have the
same root. We modified Union-Find to support an additional operation.
component\_size(i) returns the number of vertices belonging to the
component containing $i$.

Union-Find represents each component as a rooted tree, with vertices in
the graph mapping to nodes in the tree. Information about the tree is
stored in two arrays of length $N$, parent and size, which are subject
to the following invariants.

\begin{enumerate}[1.]
\item
For each node i, parent[i] = node i's parent on the tree, unless i is a
root node. If i is a root node, then parent[i] = i.

\item
Nodes i and j are in the same component if and only if they are in the
same tree, if and only if they share the same root node.

\item
If i is a root node, then size[i] = the size of the component, or the
number of nodes in the tree. If i is not a root node, then size[i] can
be anything.
\end{enumerate}

A baseline implementation of the three functions is

\begin{algorithm}
\caption{Union-Find}
\begin{algorithmic}

\Function{root}{i}
    \While{parent[i] $\neq$ i}
        \State i $\gets$ parent[i]
    \EndWhile
    \State \Return i
\EndFunction
\State 

\Function{union}{i, j}
    \State parent[root(j)] $\gets$ root(i)
\EndFunction
\State 

\Function{component\_size}{i}
    \State \Return size[root(i)]
\EndFunction

\end{algorithmic}
\end{algorithm}

In addition to the baseline code above, there are two important
optimizations:

\begin{enumerate}[1.]
\item
Weighted union maintains information of the sizes of each component so
that the root of the smaller component always becomes a child of the
larger component's root.

\item
Path compression flattens the tree with each call to root. Specifically,
when root is called on node $i$, each node traversed from $i$ to the
root has its parent set to be the root.
\end{enumerate}

With these two optimizations, the time complexity of root, union, and
component\_size has been shown to be at least as good as $O(\log^* N)$
where $\log^*$ is the iterated logarithm, defined as the number of
times the natural log must be applied to $N$ so that it becomes less
than or equal to 1.

\section{The Contractible Graph Data Structure and
Edge-Contraction Algorithm}

We propose a new data structure called the \textit{Contractible Graph}
(CG) for brain parcellation. The rationale behind the CG is a heuristic
procedure for partitioning a graph into somewhat balanced components
so as to maximize the Adjacent-Score (\ref{adjacent-score}).

The CG is a mapping of the vertices of the original graph to the
vertices of a new graph. The vertices of the CG are called
\textit{components} and between any two components there exists exactly
one weighted edge, henceforth called a \textit{link}. The weight
of a link $w_{A,B}$ between two components $A$ and $B$ in the CG equals
the average weight of all edges in the original graph between vertices
mapped to $A$ and vertices mapped to $B$. If no such edges exist,
the weight of the link is $0$. Formally,

\[ E_{A,B} = \{(i, j) \in E : i \in A, j \in B\} \]
\[ w_{A,B} = \begin{cases}
    \frac{1}{|E_{A,B}|} \sum_{(i,j) \in E_{A,B}} w_{ij} &
        \text{if } |E_{A,B}| > 0 \\
    0 & \text{otherwise}
\end{cases} \]

\includegraphics[scale = 0.5]{4_contractible_graph}

We say an edge $(i,j)$ is \textit{between} components $A$ and $B$ if
$i$ is in one of $A$ or $B$ and $j$ is in the other. The \textit{size}
of a component is the number of vertices it contains.
A \textit{contraction} of a link $(A,B)$ in a CG replaces components
$A$ and $B$ with a new component (call it $C$) containing all vertices
mapped to $A$ or $B$, as illustrated in the figure above.
Component $C$ has one link to every other component in the CG, whose
weights are the mean of the weights of the corresponding vertex edges,
or $0$ if no edge exists. Thus the contraction operation maintains the
link-invariant property of CG. This leads to the Edge-Contraction
algorithm, which begins with the original graph with all vertices as
singleton components and contracts edges in a certain order until the
graph has only $k$ components in all.

\begin{algorithm}
\caption{Edge-Contraction}
\begin{algorithmic}
\State \textbf{Input:} Undirected positive-weighted graph $G$ and
       target component number $k$
\State Create a CG from $G$ so that every vertex maps to
       a unique component
\Repeat
\State $\mathcal{S} \gets$ smallest component(s) in the CG
\State $(A,B) \gets \argmax{A \in \mathcal{S}} w(A,B)$
\State Contract $(A,B)$
\Until{CG has $k$ components}
\State \textbf{Output:} Components of CG
\end{algorithmic}
\end{algorithm}

Why does Edge-Contraction work better than the previous algorithms?
The Edge-Contraction algorithm attempts to address two problems of
the Size-Constrained Add-Edge algorithm: poor Adjacent-Score relative
to randomized graph and unbalanced parcels. We hypothesized that one
reason for a relatively low Adjacent-Score might be the following
scenario: when a vertex is added to a component, it might have multiple
edges to that component. One edge might have a very high weight; this is
the one that is officially ``added''. However, the other edges with far
lower weights are implicitly added as well, lowering the average edge
weights within the component.

The Edge-Contraction algorithm handles this issue by maintaining that
there can be at most one edge between any two components A and B, and
further that the weight on such an edge is the mean of the weights on
all edges that connect a vertex in A with a vertex in B.

\subsection{Implementation using Nested Hash Tables and Priority Queue}

In a Contractible Graph, the weight of the link between two components
depends on the summed weight of all edges between them, and the number
of such edges.

Our implementation of the CG uses \textit{nested hash tables},
diagrammed below. The outer hash table maps each component $A$ to an
inner hash table, which maps all components $B$ with a positive link to
$A$ to 1) the summed weights of the edges and 2) the number of edges
between $A$ and $B$.

\includegraphics[scale = 0.5]{4_cg_implement}

Implementing the contraction of components $B$ and $C$ into a new
component $D$ on this nested hash table requires the following steps.
The time complexity is stated assuming no hash collisions.

\begin{enumerate}
\item
Compute $\mathcal{X}$, the set of all components that either $B$ or $C$
is linked to. $O(|E_B| + |E_C|)$

\item
Create a new element in the outer hash table, $D$, and associate it
with an empty inner hash table. $O(1)$

\item
For each component $X \in \mathcal{X}$,
\begin{itemize}
\item
Retrieve $W(X,B) + W(X,C)$, the summed weights all edges between $X$
and $B$ and between $X$ and $C$, and $|E_{X,B}| + |E_{X,C}|$, the
number of such edges. These quantities are stored explicitly as a
values in the inner hash table, so this operations is $O(1)$.

\item
Add a new component name $D$ to the inner hash table of $X$ and map it
to $\big( W(X,B) + W(X,C), |E_{X,B}| + |E_{X,C}| \big)$. Delete
elements $B$ and $C$ from the inner list of $X$. $O(1)$

\item
In the $D$ inner hash table, add component name $X$ and map it to
to the same $\big( W(X,B) + W(X,C), |E_{X,B}| + |E_{X,C}| \big)$.
$O(1)$
\end{itemize}

\item
Delete $B$ and $C$ from the outer list.
\end{enumerate}

Having described the contraction step, we will next discuss how to
efficiently locate the link to be contracted.
In computer science, a \textit{Maximum Priority Queue} (MaxPQ) data
type is a set of well-ordered objects that supports the following
operations:

\begin{itemize}
\item
\textit{add(obj)}: Adds an object to the set.

\item
\textit{remove\_maximum()}: Removes and returns an object with the
largest priority in the set.
\end{itemize}

Using the heap data structure, the above two operations both run in
$O(\log n)$ time.

Each component on the CG will be associated with an element of the
priority queue. The priority of component $A$ is defined as
\[ \max_{X}\;w_{A,X} - |A| \]
Since our graph link and edge weights are all between 0 and 1, the
highest priority element in the queue always has the smallest size.
Therefore, if the priority queue is up-to-date with the CG, the
next link to be contracted according to Edge-Contraction has
an endpoint component whose priority is the highest in the queue.

However, a complication arises from the fact that a contraction can
change the priorities of components neighboring the contracting
components, thereby making the priorities stored in the MaxPQ
out-of-date. For instance, if components $A$ and $B$ are contracted, and
there is a component $C$ with positive links to both $A$ and $B$, then
the $C-A$ and $C-B$ links will be replaced by a $C-(AB)$ link with a
different weight. If either $C-A$ or $C-B$ links happened to be the
maximum-weighted links of $C$, then $C$'s priority will be lower,
and $C$ ought to be further down the queue.

To address this issue, we could re-compute the priority of every
component drawn from the MaxPQ. If the component's actual priority is
not the maximum, then it is re-inserted into the queue with updated
priority. Additionally, the maximum priority component may no longer
exist in the CG due to contraction with another component. In this case
it is simply discarded.

Without using an efficient priority queue, the linear searching method
of finding the next link to contract results in a $O\big(n (n-k)\big)$
time algorithm. Using the priority queue the time complexity of
Edge-Contraction is $O \big((n - k) (m + \log n)\big)$, where $m$ is
the average number of positive links a component has.

\subsection{Results and Extensions}

\begin{figure}
\caption{Results of Edge-Contract for Different Component Numbers}
\csvautotabular{4_edge_contract_results.csv}
\end{figure}

The Edge-Contract parcellations notably outperformed the anatomical AAL
parcellation in the Adjacency-Score. For further comparison, found the
mean edge weight in the graph to be 0.7258, which is even slightly
higher than the average adjacent within-parcel edge in AAL. This suggests
that the AAL parcellation has no connection with the functional
information contained in this fMRI data set. It shows on the other
hand that the Edge-Contract algorithm can successfully locate regions of
functional similarity.

The one apparent deficiency of Edge-Contract is the jaggedness of its
parcels. Comparison of our parcellations with the AAL shows that our
116-component parcellation -- the same number of components as AAL --
has an average parcel surface area roughly
$\big( \frac{92.14}{29.62} \big)^{\frac{2}{3} }\approx 2.11$ times
that of AAL. Visually, that difference is shown in the plots below of
a typical component from each parcellation.

%\includegraphics[scale = 1]{4_edgecontract_aal_3D}

\section{Generalized Edge-Contraction}

In the original Edge-Contraction algorithm, the criteria for selecting
the next link to contract was to search through the set of smallest
components and find the link of maximal weight. Because this criteria
takes no account of the shape of the two components to be contracted,
the resulting parcels tend to be very jagged.

To address this we expanded the criterion for finding the next link
to contract. Rather than use only the size of the component and the
weight of the link, a \textit{Generalized Edge-Contraction} algorithm
may use any piece of information stored in the Contractible Graph about
a pair of components, such as the number of edges connecting two
components. A \textit{priority function} takes information of any two
components in a CG and outputs a real number, the priority. For each
iteration, the pair of components with the largest priority is
contracted and the priorities of neighboring components with respect
to the newly conjoined component are computed.

For two components $A,B$ let $|A|$ denote size (number of vertices)
of $A$, $E_{A,B}$ denote the set of edges between $A$ and $B$, and
$w_{A,B}$ the weight of the link connecting $A$ and $B$.
The priority function of the original Edge-Contraction algorithm
is $p_0(A, B) = w_{A,B} - |B|$.

A link $(A, B)$ will have high priority if either component is small,
if the link has a large weight, and if it has a good boundary-ratio,
defined as $\frac{|E_{A,B}|}{\min(|A|,|B|)}$, which helps to minimize
jaggedness. From these notions we created a family of priority functions indexed by tunable parameters $\alpha$ and $\beta$
\[ p_1(A, B) = \frac{w_{A,B}^\alpha}{|A| + \beta}
               \frac{|E_{A,B}|}{\min(|A|,|B|)} \]
that modulate the balance of small size, large weight, and high
boundary ratio. The table below shows the results of the 116-component
and 300-component Generalized Edge-Contract parcellation when performed
for various values of $\alpha$ and $\beta$.

\begin{figure}
\caption{Results of Generalized Edge-Contract for Various Parameter Settings}
%\csvautotabular{4_gen_edge_contract_results.csv}
\end{figure}




%\chapter{Spectral Methods}
%% Need to fix optimization problems -- add spacing between min and obj
%#######################################################################

In the previous chapter, we showed how local search heuristics produced
parcels that were balanced and had high within-parcel and low
between-parcel edge weights. The central idea behind such methods
was to choose vertices to be in the same component if the edge
connecting them has high distance correlation. Vertices were added to
components one-by-one with constraints on component size, but not on
component shape. As a result, one salient issue with these
parcellations was lack of smoothness, or regularity in the parcels'
spatial shapes. There was scant resemblence between the anatomical maps
of the brain depicting smooth, rotund lobes and our jagged, web-like
parcellations.

One key reason for this phenomenon are the local search heuristics'
focus on maximizing \textit{average} within-component edge weights
(equivalently, minimizing \textit{average} between-component edge
weights because edges are either within the same component or between
different components). To get smoothness in the boundary between
components, we could either impose a penalty for too many
between-component edges and work that into the local search heuristics,
or try minimizing over the sum of all weights on between-component
edges. This chapter deals with the second approach and this family of
methods is called spectral partitioning.

Spectral partitioning constitutes the second major class of techniques
used to partition graphs. Rather than rely on local component-growing
heuristics, spectral partitioning uses information about the entire
graph at once.

Throughout this chapter, a valid partitioning
$P_k = (V_1, ..., V_k)$ of the graph $G = (V, E)$ is defined in the
same way as in chapter 3; i.e., it must satisfy

\begin{enumerate}[1.]
\item
$V_i \neq \emptyset$ for all $V_i \in \mathcal{P}_k$

\item
$\bigcup\limits_{i=1}^k V_i = V$

\item
$V_i \cap V_j = \emptyset$ for all $V_i, V_j \in \mathcal{P}_k$

\item
$V_i$ is connected (i.e. for every two vertices in $V_i$, there is a
path between them) for all $V_i \in \mathcal{P}_k$
\end{enumerate}

For all edges $(i,j) \in E$, let $w_{ij}$ denote the weight of the edge
connecting vertices $i$ and $j$.
$S^{n \times n}$ is the set of real symmetric $n \times n$ matrices.
We further define, for a given graph $G = (V, E)$, the associated

\begin{definition}
Adjacency matrix. $A \in \mathcal{S}^{n \times n}$ has entries
\[
A_{ij} = \begin{cases}
  w_{ij} & \text{if } (i,j) \in E \\
  0      & \text{otherwise} \\
\end{cases}
\]
\end{definition}

\begin{definition}
Degree matrix. $D \in \mathcal{S}^{n \times n}$
\[
D_{ij} = \begin{cases}
  \sum_{k = 1}^n A_{ik} & \text{if } i = j \\
  0                     & \text{otherwise} \\
\end{cases}
\]
\end{definition}

\section{Size-Constrained MinCut and Graph Bipartitioning}

Consider the case $k = 2$. For all $i \in V$, let $x_i = 1$ if
$i \in V_1$ and $x_1 = -1$ if $i \in V_2$. Then the sum of weights on
edges between the two components is
\begin{align*}
C(P_2)
&= \sum_{i \in V_1} \sum_{j \in V_2} A_{ij} \\
&= \sum_{i = 2}^n \sum_{j = 1}^{i-1} \frac{(x_i - x_j)^2}{4} A_{ij} 
\end{align*}
since
\[ (x_i - x_j)^2 = \begin{cases}
	4 & \mbox{if } i,j \mbox{ are in different components} \\
	0 & \mbox{otherwise}
\end{cases}\]

$C(P_2)$ can also be written in a matrix quadratic form, as

\begin{align*}
C(P_2)
&= \sum_{i = 2}^n \sum_{j = 1}^{i-1} \frac{(x_i - x_j)^2}{4} A_{ij} \\
&= \frac{1}{2} \sum_{i,j = 1}^n \frac{(x_i - x_j)^2}{4} A_{ij} \\
&= \frac{1}{2} \sum_{i,j = 1}^n
   \frac{x_i^2 + x_j^2 - 2 x_i x_j}{4} A_{ij} \\
&= \frac{1}{2} \sum_{i,j = 1}^n \frac{1 - x_i x_j}{2} A_{ij} \\
&= \frac{1}{4} \sum_{i,j = 1}^n (x_i^2 - x_i x_j) A_{ij} \\
&= \frac{1}{4} \sum_{i = 1}^n x_i^2 \sum_{j = 1}^n A_{ij}
 - \frac{1}{4} \sum_{i,j = 1}^n x_i A_{ij} x_j \\
&= \frac{1}{4} \sum_{i = 1}^n x_i^2 D_{ii} - \frac{1}{4} x^T A x \\
&= \frac{1}{4} x^T (D - A) x \\
&= \frac{1}{4} x^T L x
\end{align*}
where $L$ is called the Laplacian matrix of the graph and defined as
$L = D - A$. MinCut can thus be formulated as minimizing $x^T L x$
subject to $x \in \{-1, 1\}^n$.

Algorithms like Karger's can solve MinCut in polynomial time. However,
MinCut in this formulation lacks constraints on the size of the
partitions, and if applied to our brain parcellation problem, would
result in severely inbalanced partitions. If constraints on the sizes
of the components were added, the problem becomes NP-hard [citation].

An old but effective approach to bipartitioning uses the eigenvectors
of the Laplacian matrix and is called spectral bipartitioning.
The approach relaxes the $\{-1, 1\}$ constraint on $x$ (and rescales
$x$) so that it need only satisfy $\|x\| = 1$ ($\|\cdot\|$ here refering
to L2 norm). It is easy to see that
$\big\{ x : x \in \{-\frac{1}{\sqrt{n}}, \frac{1}{\sqrt{n}}\}^n \big\}
 \subset \big\{ x \in \R^n : \|x\| = 1 \big\}$
The problem now becomes

\begin{equation} \label{spectral_bipartition}
\begin{aligned}
\min_x      &\;& x^T L x \\
\text{s.t.} &\;& \| x \| = 1 \\
\end{aligned}
\end{equation}

Using Lagrangian multipliers, it can be shown that all optimal solutions
to the above must satisfy $L x = \lambda x$ and this problem reduces to
finding the smallest eigenvalues of $L$ and their associated
eigenvectors. In addition, \ref{Laplacian_psd} below implies that all
eigenvalues are nonnegative.

\begin{theorem} \label{Laplacian_psd}
Let $L$ be a Laplacian matrix. Then $L \succeq 0$
($L$ is positive semidefinite)

Proof. Let $x \in \R^n$. $x^T L x = x^T D x - $
\end{theorem}

Note that from the
$C(P_2) = \sum_{i > j} \frac{(x_i - x_j)^2}{4} A_{ij}
        = \frac{1}{4} x^T L x$ equivalence we know that
$0$ and $(\frac{1}{\sqrt{n}}, ..., \frac{1}{\sqrt{n}})^T$ is a minimum
eigenvalue and eigenvector to this system. For bipartitioning, the
useful eigenvector is the one that corresponds to the 2nd smallest
eigenvalue, which is nonzero if the graph as a whole is connected.
We'll denote this eigenvalue as $\lambda_1$ and corresponding unit
eigenvector as $x_1$. We have the following:

\begin{theorem}
Let $P_2$ be any valid partition into 2 components. Then
$C(P_2) \geq \lambda_1$

Proof.
\end{theorem}

In the literature, $x_1$ is often refered to as the Fiedler vector,
after the first mathematician who studied it in detail [Fiedler 1975].
From the Fiedler vector we can obtain a variety of "good" bipartitions.
We can impose a size constraint $|V_1| = s$ and obtain a bipartition
satisfying this by placing the vertices associated with the $s$ largest
entries of $x_1$ in $V_1$. This encompasses bipartitions of equal
component size. We can also sort the entries of $x_1$ and find the
largest difference between consecutive sorted entries. Vertices
corresponding to entries sorted to the left of this split can be placed
in $V_1$ and vertices sorted to the right in $V_2$. This method tends to
approximate the MinCut solution.

The result of spectral bipartitioning on a resting state fMRI scan
is shown below. As anticipated, the boundaries of between the
components are smooth.

\begin{center}
\includegraphics[scale = 0.5]{5_spectral_2_axial.png}

Axial

\includegraphics[scale = 0.5]{5_spectral_2_coronal.png}

Coronal

\includegraphics[scale = 0.5]{5_spectral_2_sagittal.png}

Sagittal
\end{center}

One can recursively apply this bipartitioning method to the component
subgraphs to obtain $k$-partitions, but there is a more elegant
approach involving additional eigenvectors that we shall discuss next.

\section{Spectral k-partitioning}


%#######################################################################


%\chapter{Symmetric Nonnegative Matrix Factorization}
%In the previous chapter, we showed that the problem of finding the
minimum ratio cut of a graph (with Laplacian matrix $L$, degree matrix
$D$, and adjacency matrix $A$) can be formulated as minimizing
\begin{equation} \label{ratio_cut}
\Tr(R^T L R)
\end{equation}
over the set, $\mathcal{R}$, of $n \times k$ matrices satisfying
\begin{enumerate}
\item
$R^T R = I$

\item
$R \geq 0$ (element-wise)

\item
$R R^T u_n = u_n$ where $u_n$ is a $n$-dimensional vector of all ones.
\end{enumerate}

If the sizes of the components in the optimal ratio cut partition
are perfectly balanced, which is equivalent to saying if the diagonal
of the optimal ratioed assignment matrix $R R^T$ has entries all
equal to $\frac{k}{n}$, then
\begin{align*}
\Tr(R^T D R) &= \sum_{i=1}^n [R R^T]_{ii} D_{ii} \\
             &= \sum_{i=1}^n \frac{k}{n} D_{ii} \\
             &= \frac{k}{n} \sum_{i,j} A_{ij}
\end{align*}
is a constant that does not depend on $R$. The same is true if each
vertex has the same degree $D_{ii} = d$, in which case
\begin{align*}
\Tr(R^T D R) &= \sum_{i=1}^n [R R^T]_{ii} D_{ii} \\
             &= d \sum_{i=1}^n [R R^T]_{ii} \\ 
             &= d k
\end{align*}
is also a constant that does not depend on $R$. In either case,
\[ \argmin{R \in \mathcal{R}} \Tr(R^T L R)
 = \argmax{R \in \mathcal{R}} \Tr(R^T A R) \]
This equality may also hold even if neither condition is true,
especially if they are approximately true.

Spectral $k$-partitioning drops the second and third constraints
of $\mathcal{R}$ to derive a closed-form minimizer of \ref{ratio_cut},
from which the original assignment matrix can be obtained by $k$-means.
This chapter deals with an alternative relaxation of $\mathcal{R}$ that
drops the first and third constraints.

\section{Symmetric Nonnegative Matrix Factorization}

For an $n \times m$ matrix $A$, a nonnegative matrix factorization
(NMF) is a pair of matrices $W \in \R^{n \times k}$ and
$H \in \R^{m \times k}$ that minimizes $\| A - W H^T \|_F^2$
subject to elementwise nonnegativity: $H \geq 0$ and $W \geq 0$.
Here, $\|X\|_F = \sqrt{\sum_{ij} X_{ij}}$ refers to the Frobenius norm.

For $n \times n$ symmetric matrices $A$, a \textit{symmetric} NMF
(SymNMF) is a matrix $H \in \R^{n \times k}$ that minimizes
$\| A - H H^T \|_F^2$, and $k$ is an arbitrary positive integer
typically much smaller than $n$.

The following theorem from \cite{Ding:05} illustrates the connection
between SymNMF and graph partitioning.

\begin{theorem}
Let $A$ be a $n \times n$ symmetric matrix. Then
\[ \argmax{H^T H = I, H \geq 0} \Tr(H^T A H)
 = \argmin{H^T H = I, H \geq 0} \| A - H H^T \|_F^2 \]

Proof. \begin{align*}
   \argmax{H^T H = I, H \geq 0} \Tr(H^T A H)
&= \argmin{H^T H = I, H \geq 0} -2 \Tr(H^T A H) \\
&= \argmin{H^T H = I, H \geq 0} \Tr(A A^T) - 2 \Tr(H^T A H)
                                + \|H^T H\|_F^2 \\
&= \argmin{H^T H = I, H \geq 0} \|A - H H^T\|_F^2
\end{align*}
\end{theorem}

If $A$ is the adjacency matrix, then under the equal vertex degrees
condition described earlier
$ \argmax{H^T H = I, H \geq 0} \Tr(H^T A H)
= \argmin{H^T H = I, H \geq 0} \Tr(H^T L H)$.
Hence an alternative approach to the minimum ratio-cut problem
is to drop the $H^T H = I$ constraint and solve the SymNMF problem:

\begin{equation} \label{sym_nmf}
\begin{aligned}
\min_{H \in \R^{n \times k}} &\;& \|A - H H^T\|_F^2 \\
\text{s.t.}                  &\;& H \geq 0          \\
\end{aligned}
\end{equation}

This relaxation has two key differences from the spectral relaxation
\label{spectral_k-partition}.
\begin{itemize}
\item
There is no closed-form solution, and the optimal value is found
via an optimization algorithm, described in the next section.

\item
The optimal assignments are recovered directly from the largest
entry in each row. There is no need for $k$-means.
\end{itemize}

\subsection{The Alternating Nonnegative Least Squares Algorithm}

\cite{Kuang:15} re-formulates \ref{sym_nmf} as a non-symmetric NMF
with a penalty on the difference between the two matrix factors:
\begin{equation} \label{nonsym_nmf}
\min_{W,H \geq 0} \|A - W H^T\|_F^2 + \alpha \|W - H\|_F^2
\end{equation}
where $W,H \in \R^{n \times k}$. The $\alpha$ parameter 

The rationale for this to use known methods for solving the
non-symmetric NMF and adapt them to the symmetric problem.
One powerful framework for solving NMF is Alternating Nonnegative
Least Squares (ANLS), which factors $A$ into nonnegative $W$ and $H$
by fixing the $H$ matrix and solving for $W$:
$$ W \gets \argmin{W \geq 0} \|A - W H^T\|_F^2 $$
and fixing this new matrix $W$ and solving for $H$:
$$ H \gets \argmin{H \geq 0} \|A - W H^T\|_F^2 $$
and repeating the two steps until convergence.
Both subproblems in the ANLS framework are convex, and the algorithm
requires only an initial $W$ to get started.

\cite{Kuang:15} describes an algorithm for solving SymNMF that uses
the ANLS framework. The objective function in \ref{nonsym_nmf} can be
re-written as
\begin{equation} \label{nls}
\norm{ \begin{bmatrix} W \\ \sqrt{\alpha} I_k \end{bmatrix} H^T
     - \begin{bmatrix} A \\ \sqrt{\alpha} W^T \end{bmatrix} }_F^2
\end{equation}
with $\begin{bmatrix} W \\ \sqrt{\alpha} I_k \end{bmatrix}$ taking on
the part of the fixed matrix and $H$ the decision matrix. The ANLS
algorithm for SymNMF is the following:

\begin{algorithm}
\caption{ANLS algorithm for SymNMF}
\begin{algorithmic}[1]
\State Initialize $H$
\Repeat
  \State $W \gets H$
  \State $H \gets \argmin{H \geq 0}
    \norm{ \begin{bmatrix} W \\ \sqrt{\alpha} I_k \end{bmatrix} H^T
         - \begin{bmatrix} A \\ \sqrt{\alpha} W^T \end{bmatrix} }_F^2$
\Until{convergence}
\end{algorithmic}  
\end{algorithm}

The $\alpha$ can be increased each iteration to force convergence of
$W$ and $H$. A recommended strategy is to multiply $alpha$ by 1.01 each
iteration.

\subsection{Block Principal Pivoting for Nonnegative Least Squares}
For details on this method, please refer to \cite{Kim:11}.

\section{Connected Partitions}

An issue with the SymNMF approach is that the partitions may not be
connected -- there may exist a vertex, none of whose neighbor vertices
belong to the same parcel that it does.

\begin{definition}[Unweighted Adjacency Matrix]
For a graph with $n$ vertices and edge set $E$, the unweighted adjacency
matrix $B \in \{0, 1\}^{n \times n}$ has entries
\[ B_{ij} = \begin{cases}
  1 & \text{if } (i,j) \in E \\
  0 & \text{otherwise}
\end{cases}\]
\end{definition}

\begin{prop} \label{connected}
Let $B$ be the unweighted adjacency matrix of a graph with $n$ vertices
and $X \in \{0, 1\}^{n \times K}$ be an assignment matrix satisfying
$X e_K = e_n$ and $X^T e_n \geq e_K$. Then the partitions of the graph
defined by $X$ are connected if and only if
\[ (B - I) X \geq 0 \]

\textit{Proof:} Let $b_i$ denote the $i$th row of $B$ and $X_k$ the
$k$th column of $X$. Since the non-zero entries of $b_i$ indicate
which vertices $i$ is neighboring and the non-zero entries of $X_k$
indicate which vertices are in the $k$th partition, the dot product
$b_i X_k$ equals the number of vertices neighboring $i$ that are in the
$k$th partition.

If the partitions of the graph are connected, then this number is
at least 1 if $k$ is the partition containing $i$. If $k$ does not
contain $i$, then $b_i X_k$ can be 0. This can be succinctly expressed
as $b_i X_k \geq X_{ik}$ for all $i = 1, ..., n$ and $k = 1, ..., K$,
which is equivalent to the matrix inequality above.
\end{prop}

This fact will be used in the next section for the problem of finding
a binary matrix factorization of $A$.


\section{Symmetric 0-1 Matrix Factorization}

The SymNMF method of graph partitioning finds continuous nonnegative
matrix $H \in R^{n \times k}$ so as to minimize $\norm{A - H H^T}_F^2$
and obtains the 0-1 assignment matrix $X$ by thresholding $H$.
In the graph partitioning case, it arguably more desirable to obtain
binary $X$ directly rather than via the continuous $H$.

This leads us to the \textit{Symmetric Binary Matrix Factorization}
problem, henceforth called SymBMF. Formally, for a symmetric matrix
$A \in [0, 1]^{n \times n}$ we want to solve
\begin{center}
\begin{tabular}{l l}
minimize   & $\norm{A - X X^T}_F^2$ \\
subject to & $X \in \{0, 1\}^{n \times k}$ \\
           & $X e_k = e_n$
\end{tabular}
\end{center}
The constraint $X^T e_n \geq 1$ can be added to ensure no column of $X$
contains all zeros. Additionally, the constraint $(B - I) X \geq 0$
(where $B$ is the unweighted adjacency matrix) can be added for graph
partitioning purposes to ensure all the partitions are connected
(\ref{connected}).

It is desirable for algorithms that solve SymBMF to work well when $A$
is sparse or incomplete. In the sparse case, when most entries of $A$
are zero, the complexity of the algorithm ideally ought to scale with
the number of non-zero entries, and not with the number of rows and
columns. Similarly in the case of incomplete $A$ there are entries of
$A$ that are unknown or that we simply don't care about approximating.
This changes the objective function from a matrix norm to a summation:
\[ \sum_{(i,j) \in A} (A_{ij} - x_i^T x_j)^2 \]
where $x_i$ denotes the $i$th row of $X$.

We present two methods that both scale well in sparse $A$ and can
handle incomplete $A$. The first is a very fast local minimizer
inspired by methods from multidimensional scaling. The second is a
mixed integer program that solves the problem globally, but is not
as efficient.

\subsection{An MDS-Inspired Method}

The central idea behind this to begin with an random $n \times k$
binary matrix $X$ that satisfies $X e_k = e_n$ and iteratively
edit each row of $X$ so as to minimize $\|A - X X^T\|_F^2$ locally.
Editing a row of $X$ here means determining which column to place the 1
in. The change made in row $i$ of $X$ only impacts the $i$th row and
$i$th column of $A - X X^T$. The column to place the 1 in that minimizes
the objective locally is:
\begin{equation} \label{col_select}
k^* = \argmin{k = 1,...,K} \|A_i - X_k\|^2
\end{equation}
where $A_i$ refers to the $i$th column of $A$ and $X_k$ to the $k$th
column of $X$. This is because the $i$th column of $X X^T$ is $X_k$,
where $k$ is the parcel that vertex $i$ has been assigned to.

In the case of incomplete matrix $A$, the column selection rule in
\ref{col_select} should replaced by
\begin{equation} \label{inc_col_select}
k^* = \argmin{k = 1,...,K} \sum_{j \in A_i} (A_{ij} - X_{jk})^2
\end{equation}
The squared terms in (\ref{col_select}) and (\ref{inc_col_select}) can
also be changed to absolute value.

\begin{algorithm}
\caption{SymBMF}
\begin{algorithmic}[1]
\State Initialize $X \in \{0, 1\}^{n \times k}$ such that $X e_k = e_n$
\State $i \gets 1$
\State $j \gets 1$
\Repeat
  \State $k \gets$ index of 1 entry of $x_i$
  \State $x_{ik} \gets 0$
  \State Compute $k^*$ by (\ref{col_select}) or (\ref{inc_col_select})
  \State $x_{ik^*} \gets 1$
  \If{$k \neq k^*$} \Comment{row $i$ has been edited}
    \State $j \gets i$
  \EndIf
  \State $i \gets (i \mod n) + 1$
\Until{$i = j$}
\end{algorithmic}
\end{algorithm}

The stopping criterion halts the loop before iteration $i$ if no rows
have been edited since $x_i$ was last edited.



\subsection{Mixed Integer Programming Method}

The MIP method begins by replacing objective's squared term in the
Frobenius norm with an L1 penalty. This new objective function is
equivalent to the original if $A$ is binary.
\begin{center}
\begin{tabular}{l l l}
minimize   & $\sum_{(i,j) \in A} |A_{ij} - x_i^T x_j|$ \\
subject to & $x_i \in \{0, 1\}^k$ & for $i = 1, ..., n$ \\
           & $e_k^T x_i = 1$ & for $i = 1, ..., n$
\end{tabular}
\end{center}
This problem would be a mixed integer program if it were not for the
quadratic $x_i^T x_j$ in the objective. We substitute a new variable
$y_{ij} = x_i^T x_j$ and find linear constraints that make this
relation true for binary $x_i$ satisfying $e_k^T x_i = 1$.
Note that an equivalent definition of
$y_{ij}$ is
\[ y_{ij} = \begin{cases}
  1 & \text{if } x_i = x_j \\
  0 & \text{otherwise}
\end{cases}\]
The following lemma provides an linearization of $x_i^T x_j$:

\begin{lemma}
Let $x_i$ and $x_j$ both be $k$-dimensional binary vectors that sum to 1.
Then $y_{ij} = x_i^T x_j$ is equivalent to
\[ y_{ij} \leq \min (x_i - x_j) + 1 \]
\[ y_{ij} \geq \max (x_i + x_j) - 1 \]

\textit{Proof:} Follows from the fact that $\min (x_i - x_j) + 1$ and
$\max (x_i + x_j) - 1$ both equal 1 if $x_i$ and $x_j$ are equal
and 0 if not.
\end{lemma}

Substituting $y_{ij}$ and adding the above linear constraints gives us
the following MIP equivalent of SymBMF.

\begin{center}
\begin{tabular}{l l l}
minimize   & $\sum_{(i,j) \in A} |A_{ij} - y_{ij}|$ \\
subject to & $x_i \in \{0, 1\}^k$ & for $i = 1, ..., n$ \\
           & $e_k^T x_i = 1$ & for $i = 1, ..., n$ \\
           & $\begin{cases}
             y_{ij} \leq x_{ik} - x_{jk} + 1 \\
             y_{ij} \geq x_{ik} + x_{jk} - 1
             \end{cases}$
           & for $(i,j) \in A$, $k = 1, ..., K$
\end{tabular}
\end{center}

To enforce partition connectedness we can add the constraint
$(B - I) X \geq 0$ from (\ref{connected}) where $B$ is the unweighted
assignment matrix.


\bibliography{ref}
\bibliographystyle{apalike}

\end{document}